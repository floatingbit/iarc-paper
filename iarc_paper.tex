\documentclass[12pt, letterpaper]{article}
\usepackage{iarc_latex_style}
\usepackage{amssymb,amsmath,listings,url,verbatim,graphicx}

\title{BeoHawk: Autonomous Quadrotor}
\begin{document}
% Keep this in mind when writing
%\begin{verbatim}
%IARC Paper Outline
%1) Abstract	5
%2) Introduction	5
%  a) Statement of the problem 
%  b) Conceptual solution to solve the problem
%    b1) Figure of overall system architecture 
%  c) Yearly Milestones
%3) Air Vehicle	15
%  a) Propulsion and Lift System 
%  b) Guidance, Nav., and Control 
%    b1) Stability Augmentation System 
%    b2) Navigation 
%    b3) Figure of control system architecture 
%  c) Flight Termination System
%4) Payload	15
%  a) Sensor Suite 
%    a1) GNC Sensors 
%    a2) Mission Sensors 
%      a21) Target Identification 
%      a22) Threat Avoidance 
%  b) Communications 
%  c) Power Management System 
%  d) Sub-Vehicle (if any)
%5) Operations	10
%  a) Flight Preparations 
%    a1) Checklist(s) 
%  b) Man/Machine Interface
%6) Risk Reduction	15
%  a) Vehicle Status 
%    a1) Shock/Vibration Isolation 
%    a2) EMI/RFI Solutions 
%  b) Safety 
%  c) Modeling and Simulation 
%  d) Testing
%7) Conclusion	5 
%8) References	5
%Note paragraph numbers if referencing former papers
%9) Overall Format, Completeness, and Readability 25 
%TOTAL	100
%\end{verbatim}
%\break

% The real paper starts here
\maketitle
\begin{people}
\name{Christopher Li}
\org{University of Southern California}
\name{Rustom Jehangir}
\org{University of Southern California}
\end{people}

%1) Abstract	5
\begin{abstract}
	In this paper, we introduce a Micro UAV system that can explore an unknown indoor space without the assistance of a positioning system such as GPS. The robot takes in various kind of sensing measurements, handles them with probabilistic theories, and completes tasks such as stabilization and SLAM. (5 points)
\end{abstract}

%2) Introduction	5
%  a) Statement of the problem 
%  b) Conceptual solution to solve the problem
%    b1) Figure of overall system architecture 
%  c) Yearly Milestones
\section{Introduction (5)}
The USC Aerial Robotics Team's \textit{Beohawk} quadcopter was designed to suit the requirements of the International Aerial Robotics Competition. The quadcopter uses a variety of sensors to measure and identify the environment it is in with the effort of searching for small object in a complex and unknown setting.

\subsection{Statement of the Problem}
The quadcopter must navigate a large, unknown environment to find a small object, retrieve it, and return, all while avoiding several potential hazards and completing the mission in a short period of time. The challenge in developing a robot that can solve such is problem is to recognize the environment based on the relatively inaccurate measurements provided by sensors and using that information to make decisions and take actions. 

\subsection{Conceptual Solution to Solve the Problem}
The USC Aerial Robotics Team is implementing a solution based on a traditional approach to robotics with modifications for the mission and for the mechanics of the robot. The quadcopter is custom built to be within the required weight and size range but uses a commercial control board for low level control. An onboard computer uses the Robotic Operating System (ROS) and takes advantage of many of the packages and tools made available by public contributors. As a young team, much of the development is trial and error with much learning along the way.

\subsubsection{Figure of Overall System Architecture}

Figure \eqref{fig:architecture} shows the basic system architecture of the quadcopter. The quadcopter's low level control including stability, attitude control, altitude control, and position control are all performed by the low-level control board. This board is an Arduino based board that has sensors and motor outputs. It also receive radio-control signals and allows control to be over-ridden by a human pilot.

The main computer is a small form-factor x86 based computer that handles higher level control. This computer is running ROS, which performs optical flow calculations for basic positional stability. This computer also runs the navigator that decides where the robot will move and what actions it will take. More computationally expensive processes such as SLAM, take place on an off-board computer with a more powerful processor and more memory.

\begin{figure}[h]
\centering
\includegraphics[width=12cm]{Architecture-Diagram.png}
\caption{General architecture of the Beohawk control system. [[[This needs to be updated but serves as a placeholder]]].} 
\label{fig:architecture}
\end{figure}

\subsection{Yearly Milestones}
Not sure what's supposed to go in here?


%3) Air Vehicle	15
%  a) Propulsion and Lift System 
%  b) Guidance, Nav., and Control 
%    b1) Stability Augmentation System 
%    b2) Navigation 
%    b3) Figure of control system architecture 
%  c) Flight Termination System
\section{Air Vehicle (15)}
All pages have 2.5cm (1 inch) margins on all sides. Only page numbers are allowed to violate margins. Papers are limited to 12 pages (including figures and references, if any). The format shall be single-sided with text occupying a space no greater than 22.86 cm (9 inches) tall by 16.51 cm (6.5 inches) wide centered on each page.

\subsection{Propulsion and Lift System}
All pages have 2.5cm (1 inch) margins on all sides. Only page numbers are allowed to violate margins. Papers are limited to 12 pages (including figures and references, if any). The format shall be single-sided with text occupying a space no greater than 22.86 cm (9 inches) tall by 16.51 cm (6.5 inches) wide centered on each page.

\subsection{Guidance, Navigation, and Control}
All pages have 2.5cm (1 inch) margins on all sides. Only page numbers are allowed to violate margins. Papers are limited to 12 pages (including figures and references, if any). The format shall be single-sided with text occupying a space no greater than 22.86 cm (9 inches) tall by 16.51 cm (6.5 inches) wide centered on each page.

\subsubsection{Stability Augmentation System}
All pages have 2.5cm (1 inch) margins on all sides. Only page numbers are allowed to violate margins. Papers are limited to 12 pages (including figures and references, if any). The format shall be single-sided with text occupying a space no greater than 22.86 cm (9 inches) tall by 16.51 cm (6.5 inches) wide centered on each page.

\subsubsection{Navigation}
All pages have 2.5cm (1 inch) margins on all sides. Only page numbers are allowed to violate margins. Papers are limited to 12 pages (including figures and references, if any). The format shall be single-sided with text occupying a space no greater than 22.86 cm (9 inches) tall by 16.51 cm (6.5 inches) wide centered on each page.

\subsubsection{Figure of Control System Architecture}
All pages have 2.5cm (1 inch) margins on all sides. Only page numbers are allowed to violate margins. Papers are limited to 12 pages (including figures and references, if any). The format shall be single-sided with text occupying a space no greater than 22.86 cm (9 inches) tall by 16.51 cm (6.5 inches) wide centered on each page.

\subsection{Flight Termination System}
All pages have 2.5cm (1 inch) margins on all sides. Only page numbers are allowed to violate margins. Papers are limited to 12 pages (including figures and references, if any). The format shall be single-sided with text occupying a space no greater than 22.86 cm (9 inches) tall by 16.51 cm (6.5 inches) wide centered on each page.


%4) Payload	15
%  a) Sensor Suite 
%    a1) GNC Sensors 
%    a2) Mission Sensors 
%      a21) Target Identification 
%      a22) Threat Avoidance 
%  b) Communications 
%  c) Power Management System 
%  d) Sub-Vehicle (if any)
\section{Payload (15)}
All pages have 2.5cm (1 inch) margins on all sides. Only page numbers are allowed to violate margins. Papers are limited to 12 pages (including figures and references, if any). The format shall be single-sided with text occupying a space no greater than 22.86 cm (9 inches) tall by 16.51 cm (6.5 inches) wide centered on each page.

\subsection{Sensor Suite}
All pages have 2.5cm (1 inch) margins on all sides. Only page numbers are allowed to violate margins. Papers are limited to 12 pages (including figures and references, if any). The format shall be single-sided with text occupying a space no greater than 22.86 cm (9 inches) tall by 16.51 cm (6.5 inches) wide centered on each page.

\subsubsection{GNC Sensors}
All pages have 2.5cm (1 inch) margins on all sides. Only page numbers are allowed to violate margins. Papers are limited to 12 pages (including figures and references, if any). The format shall be single-sided with text occupying a space no greater than 22.86 cm (9 inches) tall by 16.51 cm (6.5 inches) wide centered on each page.

\subsubsection{Mission Sensors}
All pages have 2.5cm (1 inch) margins on all sides. Only page numbers are allowed to violate margins. Papers are limited to 12 pages (including figures and references, if any). The format shall be single-sided with text occupying a space no greater than 22.86 cm (9 inches) tall by 16.51 cm (6.5 inches) wide centered on each page.

\paragraph{Target Identification}
All pages have 2.5cm (1 inch) margins on all sides. Only page numbers are allowed to violate margins. Papers are limited to 12 pages (including figures and references, if any). The format shall be single-sided with text occupying a space no greater than 22.86 cm (9 inches) tall by 16.51 cm (6.5 inches) wide centered on each page.

\paragraph{Threat Avoidance}
All pages have 2.5cm (1 inch) margins on all sides. Only page numbers are allowed to violate margins. Papers are limited to 12 pages (including figures and references, if any). The format shall be single-sided with text occupying a space no greater than 22.86 cm (9 inches) tall by 16.51 cm (6.5 inches) wide centered on each page.

\subsection{Communications}
All pages have 2.5cm (1 inch) margins on all sides. Only page numbers are allowed to violate margins. Papers are limited to 12 pages (including figures and references, if any). The format shall be single-sided with text occupying a space no greater than 22.86 cm (9 inches) tall by 16.51 cm (6.5 inches) wide centered on each page.

\subsection{Power Management System}
All pages have 2.5cm (1 inch) margins on all sides. Only page numbers are allowed to violate margins. Papers are limited to 12 pages (including figures and references, if any). The format shall be single-sided with text occupying a space no greater than 22.86 cm (9 inches) tall by 16.51 cm (6.5 inches) wide centered on each page.

\subsection{Sub-Vehicle (we won't have one)}
All pages have 2.5cm (1 inch) margins on all sides. Only page numbers are allowed to violate margins. Papers are limited to 12 pages (including figures and references, if any). The format shall be single-sided with text occupying a space no greater than 22.86 cm (9 inches) tall by 16.51 cm (6.5 inches) wide centered on each page.


%5) Operations	10
%  a) Flight Preparations 
%    a1) Checklist(s) 
%  b) Man/Machine Interface
\section{Operations (10)}
All pages have 2.5cm (1 inch) margins on all sides. Only page numbers are allowed to violate margins. Papers are limited to 12 pages (including figures and references, if any). The format shall be single-sided with text occupying a space no greater than 22.86 cm (9 inches) tall by 16.51 cm (6.5 inches) wide centered on each page.

\subsection{Flight Preparations}
All pages have 2.5cm (1 inch) margins on all sides. Only page numbers are allowed to violate margins. Papers are limited to 12 pages (including figures and references, if any). The format shall be single-sided with text occupying a space no greater than 22.86 cm (9 inches) tall by 16.51 cm (6.5 inches) wide centered on each page.

\subsubsection{Checklists}
All pages have 2.5cm (1 inch) margins on all sides. Only page numbers are allowed to violate margins. Papers are limited to 12 pages (including figures and references, if any). The format shall be single-sided with text occupying a space no greater than 22.86 cm (9 inches) tall by 16.51 cm (6.5 inches) wide centered on each page.

\subsection{Man/Machine Interface}
All pages have 2.5cm (1 inch) margins on all sides. Only page numbers are allowed to violate margins. Papers are limited to 12 pages (including figures and references, if any). The format shall be single-sided with text occupying a space no greater than 22.86 cm (9 inches) tall by 16.51 cm (6.5 inches) wide centered on each page.


%6) Risk Reduction	15
%  a) Vehicle Status 
%    a1) Shock/Vibration Isolation 
%    a2) EMI/RFI Solutions 
%  b) Safety 
%  c) Modeling and Simulation 
%  d) Testing
\section{Risk Reduction (15)}
Use tables and figures to concisely state your point. A table title appears above the table it references and appears in all caps and centered, whereas a figure title appears beneath the figure and with only leading capitalization. Both table and figure titles are italicized as shown in the examples below:

\subsection{Vehicle Status}
Use tables and figures to concisely state your point. A table title appears above the table it references and appears in all caps and centered, whereas a figure title appears beneath the figure and with only leading capitalization. Both table and figure titles are italicized as shown in the examples below:

\subsubsection{Shock/Vibration Isolation}
Use tables and figures to concisely state your point. A table title appears above the table it references and appears in all caps and centered, whereas a figure title appears beneath the figure and with only leading capitalization. Both table and figure titles are italicized as shown in the examples below:

\subsubsection{EMI/RFI Solutions}
Use tables and figures to concisely state your point. A table title appears above the table it references and appears in all caps and centered, whereas a figure title appears beneath the figure and with only leading capitalization. Both table and figure titles are italicized as shown in the examples below:

\subsection{Safety}
Use tables and figures to concisely state your point. A table title appears above the table it references and appears in all caps and centered, whereas a figure title appears beneath the figure and with only leading capitalization. Both table and figure titles are italicized as shown in the examples below:

\subsection{Modeling and Simulation}
Use tables and figures to concisely state your point. A table title appears above the table it references and appears in all caps and centered, whereas a figure title appears beneath the figure and with only leading capitalization. Both table and figure titles are italicized as shown in the examples below:

\subsection{Testing}
Use tables and figures to concisely state your point. A table title appears above the table it references and appears in all caps and centered, whereas a figure title appears beneath the figure and with only leading capitalization. Both table and figure titles are italicized as shown in the examples below:

\begin{table}[h]
\centering
\begin{tabular}{l  r  r  r}
                                       & Used  & Avail. & Perc. \\
  Number of Slices:                    &  684  & 4656  &  14\%  \\
  Number of Slice Flip Flops:          &  198  & 9312  &   2\%  \\
  Number of 4 input LUTs:              & 1316  & 9312  &  14\%  \\
  Number of IOs:                       &   37  &       &      \\
  Number of bonded IOBs:               &   36  &  232  &  15\%  \\
  Number of BRAMs:                     &    2  &   20  &  10\%  \\
  Number of MULT18X18SIOs:             &   10  &   20  &  50\%  \\
  Number of GCLKs:                     &    2  &   24  &   8\%  \\
\end{tabular}
\caption{Resource usage.}
\label{tab:usage}
\end{table}


%7) Conclusion	5 
\section{Conclusion (5)}
Use tables and figures to concisely state your point. A table title appears above the table it references and appears in all caps and centered, whereas a figure title appears beneath the figure and with only leading capitalization. Both table and figure titles are italicized as shown in the examples below:


%8) References	5
\bibliographystyle{IEEEbib}
\begin{thebibliography}{10}
\bibitem[1]{bib:guitarpitchshifter} \url{http://guitarpitchshifter.com/}
\bibitem[2]{bib:mathworldfft} \url{http://mathworld.wolfram.com/FastFourierTransform.html}
\bibitem[3]{bib:ctdft} 	Cooley, J. W. and J. W. Tukey, ``An Algorithm for the Machine Computation of the Complex Fourier Series," Mathematics of Computation, Vol. 19, April 1965, pp. 297-301.
\bibitem[4]{bib:butterfly} Takala, J. and K. Punkka, ``Butterfly Unit Supporting Radix-4 and Radix-2 FFT," \url{http://ticsp.cs.tut.fi/images/4/48/Cr1028-riga.pdf}
\bibitem[5]{bib:fftdit} \url{http://cnx.org/content/m12016/latest/}
\bibitem[6]{bib:ffttricks} \url{http://cnx.org/content/m12021/latest/}
\end{thebibliography}


\end{document}